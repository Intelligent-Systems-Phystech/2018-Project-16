\documentclass[12pt,twoside]{article}
\usepackage{jmlda}

\begin{document}
\title
    {Оценка оптимального объёма выборки для задач классификации}
\author
    {Харатян~А.\,С., Катруца~А.\,М.$^1$, Стрижов~В.\,В.$^2$} % основной список авторов, выводимый в оглавление
\email
	{haratyan.as@phystech.edu; aleksandr.katrutsa@phystech.edu; strijov@phystech.edu}

\thanks
    {Работа выполнена при финансовой поддержке РФФИ, проект \No\,00-00-00000.
     Научный руководитель:  Стрижов~В.\,В.
     Консультант:  Катруца~А.\,М.}

\organization
    {$^1$Московский физико-технический институт, Москва, Россия;$^2$Вычислительный центр им. А. А. Дородницына ФИЦ ИУ РАН, Москва, Россия}
    
\abstract
	{В данной статье рассматривается задача подбора оптимального числа объектов выборки для их классификации. Проблема находит применение в ряде научных областей. Например, в медицине постановка диагноза болезни пациентам основывается на результатах анализов биомедицинских данных, получение которых  либо несёт много расходов, либо крайне затруднительно. В таком случае необходимо подобрать оптимальное количество образцов, достаточного для обеспечения необходимой точности постановки диагноза. При увеличении этого числа процент правильной классификации объектов меняется незначительно, а при уменьшении заметно сокращается. В качестве другого примера можно привести число подопытных животных, необходимых для проверки научных гипотез. В этом случае также требуется минимизировать количество образцов данных для достижения достаточной точности. В работе рассмотрены различные критерии и показывается, применение каких из них выявляет наилучшее качество классификации.

\bigskip
\noindent
\textbf{Ключевые слова}: \emph {определение оптимального объёма выборки}.

}



\maketitle



\end{document}
\grid
