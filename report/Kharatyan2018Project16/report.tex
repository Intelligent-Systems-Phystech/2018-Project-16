\documentclass[12pt,twoside]{article}
\usepackage{jmlda}

\begin{document}
\title
    {Оценка оптимального объёма выборки для задач классификации}
\author
    {Харатян~А.\,С., Катруца~А.\,М.$^1$, Стрижов~В.\,В.$^2$} % основной список авторов, выводимый в оглавление
\email
	{haratyan.as@phystech.edu; aleksandr.katrutsa@phystech.edu; strijov@phystech.edu}

\thanks
    {Работа выполнена при финансовой поддержке РФФИ, проект \No\,00-00-00000.
     Научный руководитель:  Стрижов~В.\,В.
     Консультант:  Катруца~А.\,М.}

\organization
    {$^1$Московский физико-технический институт, Москва, Россия;$^2$Вычислительный центр им. А. А. Дородницына ФИЦ ИУ РАН, Москва, Россия}
    
\abstract
	{В статье рассматривается задача выбора оптимального числа объектов выборки для их классификации. Исследуется использование порождающих и разделяющих вероятностных моделей бинарной классификации. Обсуждается проблема медицинской диагностики пациентов. Определяется понятие достаточности объёма выборки. Показывается, какими методами возможно выбрать оптимальное количество объектов, обеспечивающее необходимую точность классификации объектов. В работе рассматривается, применение каких критериев выявляет наилучшее качество классификации. Приводится теоретическое и практическое обоснование предложенных критериев.

\bigskip
\noindent
\textbf{Ключевые слова}: \emph {определение оптимального объёма выборки}.

}



\maketitle



\end{document}
\grid
