\documentclass[12pt,twoside]{article}
\usepackage{jmlda}

\begin{document}
\title
    {Оценка оптимального объёма выборки для задач классификации}
\author
    {Харатян~А.\,С., Катруца~А.\,М.$^1$, Стрижов~В.\,В.$^2$} % основной список авторов, выводимый в оглавление
\email
	{haratyan.as@phystech.edu; aleksandr.katrutsa@phystech.edu; strijov@phystech.edu}

\thanks
    {Работа выполнена при финансовой поддержке РФФИ, проект \No\,00-00-00000.
     Научный руководитель:  Стрижов~В.\,В.
     Консультант:  Катруца~А.\,М.}

\organization
    {$^1$Московский физико-технический институт, Москва, Россия;$^2$Вычислительный центр им. А. А. Дородницына ФИЦ ИУ РАН, Москва, Россия}
    
\abstract
	{В статье рассматривается задача выбора оптимального числа объектов выборки для их классификации. Исследуется использование порождающих и разделяющих вероятностных моделей бинарной классификации. Обсуждается проблема медицинской диагностики пациентов. Определяется понятие достаточности объёма выборки. Показывается, какими методами возможно выбрать оптимальное количество объектов, обеспечивающее необходимую точность классификации объектов . В работе рассматривается, применение каких критериев выявляет наилучшее качество классификации. Приводится теоретическое и практическое обоснование предложенных критериев. Используется модель логистической регрессии.

\bigskip
\noindent
\textbf{Ключевые слова}: \emph {определение оптимального объёма выборки, логистическая регрессия, расстояние Кульбака-Лейблера}.

}


\maketitle


\section{Введение}

Работа посвящена оценке оптимального объёма исследуемой выборки применительно к проблемам медицинской диагностики. Рассматриваются биомедицинские данные пациентов как выборка. Каждый пациент набором признаков. 
Получение данных о пациентах требует немалых средст. В случае, если количество данных избыточно, то их измерения приносят крайне неоправданные расходы. В связи с этим поднимается вопрос оптимального количества измерений.
Ввиду дороговизны анализов всех признаков оценка измерений должна быть точной. 

Для нахождения оценки используется модель логистической регрессии\cite{hosmer2013applied}. Стандартной практикой является использование статистических методов\cite{demidenko2007sample} для оценивания объёма данных при помощи логистической регрессии. Введём понятие устойчивости модели в отношении объёма выборки. Будем называть модель устойчивой, если при изменении малом изменении объёма параметры модели меняются незначительно.
Если размер выборки крайне мал и недостаточен, то параметры модели меняются скачкообразно при увеличении объёма.Соответственно, увеличивая объём выборки, мы повышаем устойчивость модели. В качестве показателя устойчивости
для моделей будем использовать расстояние Кульбака-Лейблера. Для того чтобы показать отличие моделей в устойчивости будем использовать разность усредненных значений расстояний К-Л, вычисленных на разных выборках одного и того же объёма. Если объекты порождены одинаковым распределением, то при росте объёма выборок разность расстояний К-Л между моделями падает. Достигнув необходимого показателя устойчивости, можно легко вычислить оптимальный размер данных.

Для вычислительного экперимента используются реальные данные 569 пациентов с 30 признаками и метками об опухоли молочной железы: доброкачественная или злокачественная. Как описано выше, будем обучать модели на разных подвыборках и после достижения необходимого уровня устойчивости получим оптимальный объём данных.

\bibliographystyle{jmlda-rus}
\bibliography{literature}
\end{document}
\grid
