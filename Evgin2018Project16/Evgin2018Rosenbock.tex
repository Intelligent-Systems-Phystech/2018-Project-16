\documentclass[12pt,twoside]{article}
\usepackage{jmlda}
\usepackage{cite}
\title
    [Изучение методов М.О. на примере задачи Rosenbrock]
    {Изучение методов М.О. на примере задачи Rosenbrock}
\author
    {Евгин~А.\,А.$^1$} % основной список авторов, выводимый в оглавление
\thanks
    {Научный руководитель:  Стрижов~В.\,В.
   Задачу поставил:  Катруца~А.\,М.
    Консультант:  T. Гадаев}
\email
    {aleasims@gmail.com}
\organization
     {$^1$Московский физико-технический институт}

\abstract
    {Задача посвящена функции Розенброка, хорошей функции многих переменных.

\bigskip
\textbf{Ключевые слова}: \emph {функция Розенброка, машинное обучение, tensor flow}

\begin{document}
\maketitle

\section{Введение}
\paragraph{Цель исследования:}
Изучить различные методы оптимизации.
\paragraph{Предмет исследования:}
Функция Розенброка
\paragraph{Решаемая в данной работе задача:}
В данной работе основное внимание уделяется различным подходам к оптимизации.

\paragraph{Предлагаемое решение:}
Протестировать в рамках tensorflow: SGD, Adam и метод Ньютона

\subsection{Постановка задачи}
\paragraph{Функция и модель:}
Тренировочные данные будем генерировать функцией Розенброка, которую часто используют в качестве эталонного теста алгоритмов оптимизации:
\begin{equation}
f(x, y)=(a-x)^2+b(y-x^2)^2
\end{equation}
Чем же она хороша?
1) Красивый график.
2) Глобальный минимум находится внутри длинной, узкой, параболической плоской долины. Найти долину тривиально, а глобальный минимум сложно.
3) Есть многомерный вариант. Придумать хорошую функцию многих переменных не так уж и просто.

\subsection{Вывод}
Различные методы оптимизации изучены и сравнены.

\bibliographystyle{plain}
\bibliography{Evgin2018Rosenbrock}
\nocite{*}
\end{document}
